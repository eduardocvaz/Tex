\documentclass{article}
\usepackage[utf8]{inputenc}

\author{Dawerton Eduardo Carlos Vaz }

\def \adivb{{$a\ |\ b$}}  
\def \bdivc{{$b\ |\ c$}}  
\def \adivc{{$a\ |\ c$}} 
\def \aexpb{{$a\ \| \ b$}}  
\def \bexpc{{$b\ \| \ c$}}  
\def \aexpc{{$a\ \| \ c$}} 


\usepackage{bbding}
\usepackage{natbib}
\usepackage{amsmath}

\begin{document}


\section{Definição D2.}
Sejam $a, b \in \mathbf Z _{\ge 0}$. Dizemos que a explode b \ (\aexpb) ,sse existe inteiro $k \ge 0$ tal que $a^k=\ b$.

\section{Proposição P1.}

Para quaisquer $a,b \in \mathbf Z _{\ge 0}$, se \aexpb , então \adivb.
%.Demonstração

\section{Demonstração}

Vou demonstrar a proposição.\newline
Sejam $a,b \in \mathbf Z _{\ge 0}$ tais que \aexpb.\ Logo seja $k \in \mathbf Z _{\ge 0} $ tal que $a^k=b$.\newline

Dividindo em Casos.\newline

CASO $k=0$. Calculamos:\newline



\begin{align*}
a^k& =
&& \text{}\\
& =a^0
&& \text{}\\
& = 1
&& \text{(Propriedade da potenciação)}\\
& = 1\cdot1
&& \text{}\\
& = b
&& \text{}\\
\end{align*}

Como $1 \in \mathbf Z$, logo \adivb. \newline

CASO $k>0$. Calculamos:




\begin{align*}
a^k& =
&& \text{}\\
& =a \cdot (a^{k-1})
&& \text{(Propriedade da potenciação)}\\
\end{align*}

Como $(a^{k-1}) \in \mathbf Z$, logo \adivb. \RectangleBold

%.FimDemonstração

 \newpage

 
\section{Definição D1.}
Sejam $a, b \in \mathbf Z _{\ge 0}$. Dizemos que a explode b \ (\aexpb) \ sse existe inteiro $k$ tal que $a^k=\ b$.

\section{Proposição P2.}

Para quaisquer $a,b,c \in \mathbf Z _{\ge 0}$, se \aexpb \ e  \ \bexpc , então \aexpc{}.

%.Demonstração

\section{Demonstração}

Vou demonstrar a proposição.\newline
Sejam $a,b,c \in \mathbf Z _{\ge 0}$ tais que \aexpb \ \ e \ \bexpc.\ Logo sejam $w,k \in \mathbf Z $ tais que $a^w=b$ e $b^k=c$.\newline

Calculamos:\newline



\begin{align*}
b^k& =
&& \text{}\\
& ={(a^w)}^k
&& \text{($b=a^w$)}\\
& = {a}^{(w \cdot k)}
&& \text{(Propriedade da potenciação)}\\
& = c.
&& \text{}\\
\end{align*}

Como ${(w \cdot k)} \in \mathbf Z$, logo \aexpc.  \RectangleBold

%.FimDemonstração
 \newpage

 
\section{Definição D3.}
Sejam $a, b \in \mathbf Z _{\ge 0}$. Dizemos que a explode b \ (\aexpb) ,sse existe inteiro $k > 0$ tal que $a^k=\ b$.

\section{Proposição P3.}

Para quaisquer $a,b,c \in \mathbf Z _{\ge 0}$, se \adivb \ \ e \ \bexpc \ então \adivc.

%.Demonstração

\section{Demonstração}

Vou demonstrar a proposição.\newline
Sejam $a,b,c \in \mathbf Z _{\ge 0}$ tais que \adivb \ \ e \ \bexpc.\ Logo sejam $w,k \in \mathbf Z  _{>0}$ tais que $a \cdot w=b$ e $b^k=c$.\newline

Calculamos:\newline



\begin{align*}
b^k& =
&& \text{}\\
& ={(a\cdot w)}^k
&& \text{($b=a\cdot w$)}\\
& = {a^k \cdot w^k}
&& \text{(Propriedade da potenciação)}\\
& = a\cdot {(a^{k-1}) \cdot w^k}
&& \text{(Propriedade da potenciação, aplicada em $a^k$)}\\
& = c.
&& \text{}\\
\end{align*}

Como ${(a^{k-1} \cdot w^k)} \in \mathbf Z$, logo \adivc.  \RectangleBold

%.FimDemonstração

\end{document}
