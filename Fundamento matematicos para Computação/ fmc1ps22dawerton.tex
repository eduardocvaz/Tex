\documentclass[a4paper]{article}
\usepackage[utf8]{inputenc}

\author{Dawerton Eduardo Carlos Vaz }


\usepackage{bbding}
\usepackage{natbib}
\usepackage{amsmath}
\usepackage{ amssymb}
\usepackage[T1]{fontenc}
\usepackage{fancyhdr}
\usepackage[pdftex]{hyperref}

\def\ints{{\mathbb Z}}
\def\nats{{\mathbb N}}
\def\by#1{&& \text{(#1)}\\}
\def\halmos{ \begin{flushright} \RectangleBold \end{flushright}}

\begin{document}


\section*{Problema 1.}
Demonstre que dado qualquer inteiro a, existem únicos inteiros q e r tais que $a = 3q + r$ e $-1 \leq r \leq 1$.



%demonstração
\section*{Demonstração}
Existência\\\\
Seja $a \in \ints{}$\\\\
Dividindo em casos\\\\
Caso $3 \nmid a$ \\
Logo existe $q \in \ints$ tal que $a=3q+1$ ou $a=3q-1$\\
Logo $a=3q+r$ com $r=1 $ ou $ r=-1$\\\\
Caso $3 \mid a$ \\
Logo existe $q' \in \ints$ tal que $a=3q'+0$\\
Logo $a=3q'+r'$ com $r'=0$\\\\
Logo pelos 2 casos existe $q\in \ints$ tal que $a=3q+r$ com $r=1$ ou $r=-1$ ou $r=0$\\
Logo existe $q,r \in \ints$ tais que $a=3q+r$ $\And$  $-1 \leq r \leq 1$
 \halmos
 Unicidade\\\\
 Suponha $q,r,q',r' \in \ints$ tais  que  \\\\
 $a=3q +r$ $\And$  $-1 \leq r \leq 1$ $\And$\\
 $a=3q'+r'$ $\And$  $-1 \leq r' \leq 1$\\\\
Logo $3q'+r'= 3q+r$\\
Logo $r-r'= 3(q-q')$\\
Como $r<3$\\
logo $r-r'=0$\\
logo $q-q'=0$\\
logo $q=q'\And r=r'$\\
\halmos
%fimdemonstração
 \newpage
 \section*{Problema 2.}
\href{https://piazza.com/class_profile/get_resource/keqeithjxj94mi/khq0si0s95z71}{Clique aqui}


%demonstração
\section*{Demonstração} 
Existência\\
Vou demonstrar para $x \in \ints$.\\
Logo que multiplicando $d_m...d_0$ por $-1$ individualmente temos o valor negativo da representação.
Logo demonstrano para $x \in \ints$\\\\
Pela demonstração do problema 1, Sejam $q,r$ tais que\\
$x=3q +r$ $\And$  $-1 \leq r \leq 1$\\
Usando indução forte\\
Como $q<x$\\
Logo pela H.I\\
$q=d_m3^m+...+d_13^1+d_03^0$\\
Logo $x=3(d_m3^m+...+d_13^1+d_03^0)+r$\\
Aplicando a propriedade distributiva\\
Logo $x=d_m3^{m+1}+...+d_13^2+d_03^1+r$\\
Logo $x=d_m3^{m+1}+...+d_13^2+d_03^1+r3^0$\\
onde $-1 \leq r \leq 1$\\
\halmos
Unicidade\\
A unicidade é dada como consequência imediata da unicidade da demonstração do problema 1 e da Hipotese indutiva.
 \newpage
%fimdemonstração

\section*{Problema 3.}
Demonstre por indução que para todo $n \in \nats$\\
\begin{align*}
\sum_{i=0}^{n}i\cdot i!& = (n + 1)! - 1     &&    \\
\end{align*}
%demonstração
\section*{Demonstração}

Vou demonstrar a proposição.\newline
Por indução!

BASE $0\cdot 0!& \stackrel{?}{=} (0 + 1)! - 1$\\
\begin{align*}
0\cdot 0!& = 0     &&    \\
(0 + 1)! - 1& = 0     &&    \\
\end{align*}
Passo Indutivo. Seja $k\in \nats$ tal que:\\

\begin{align*}
\sum_{i=0}^{k}i\cdot i!& = (k + 1)! - 1     \by{H.I}
\end{align*}
Basta demonstrar que $\sum_{i=0}^{k+1}i\cdot i!& = ((k+1) + 1)! - 1$\\ \\
Calculamos
\begin{align*}
\sum_{i=0}^{k+1}i\cdot i!& = \sum_{i=0}^{k}i\cdot i!+(k+1)(k+1)!     &&    \\
& = (k+1)!-1+(k+1)(k+1)!     \by{H.I}
& =(k+1)!+(k+1)(k+1)!-1    &&   \\
& =(k+1)!+\underbrace{(k+1)!+(k+1)!+...+(k+1)!}_{(k+1) \text{ vezes}}-1    &&   \\
& =\underbrace{(k+1)!+(k+1)!+(k+1)!+...+(k+1)!}_{(k+2) \text{ vezes}}-1    &&   \\
& =(k+2)\cdot(k+1)!-1    &&   \\
& =((k+1)+1)!-1    &&   \\
\end{align*}
 \halmos
%fimdemonstração

\newpage
\section*{Problema 4.}
Dawerton 4 (demonstração errada e sem tempo para consertar devido à outras turmas)




%fimdemonstração
\pagestyle{fancy}.
\rfoot{Só isso mesmo.}
\end{document}
