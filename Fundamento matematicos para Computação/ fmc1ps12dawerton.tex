\documentclass[a4paper]{article}
\usepackage[utf8]{inputenc}

\author{Dawerton Eduardo Carlos Vaz }



\usepackage{bbding}
\usepackage{natbib}
\usepackage{amsmath}
\usepackage{ amssymb}
\usepackage[T1]{fontenc}
\usepackage{fancyhdr}

\def\ints{{\mathbb Z}}
\def\nats{{\mathbb N}}
\def\by#1{&& \text{(#1)}\\}
\def\halmos{ \begin{flushright} \RectangleBold \end{flushright}}

\begin{document}


\section*{Problema 1.}
Com as definições das operações do capítulo  \guillemotleft Naturais; recursão; indução\guillemotright , demonstre a distributividade
esquerda da $\cdot$ sobre a $+$ nos naturais:



 \begin{center} $(\forall x)(\forall a)(\forall b)[x\cdot(a+b)=(x\cdot a)(x\cdot b)]$ \end{center}

Das propriedades de operações podes apenas considerar como dada a associatividade da $+$

%demonstração
\section*{Demonstração}

Vou demonstrar a proposição.\newline
Por indução no $b$!

BASE: $x\cdot(a+0)\stackrel{?}{=}(x\cdot a)+(x\cdot 0) $ 

Calculamos:


\begin{align*}
x\cdot (a+0)& = x\cdot a            \by {Pela (a1) com $n:=a$} \\
(x\cdot a)+(x\cdot0)& =x\cdot a+0   \by {Pela (m1) com $n:=x$}
& = x\cdot a                        \by {Pela (a1) com $n:=a$}
\end{align*}

Passo Indutivo. Seja $k\in \ints$ tal que:

\begin{center}$x\cdot(a+k)=(x\cdot a)+(x\cdot k)$  (H.I)\end{center}

Basta demonstrar que $x\cdot(a+(k+1))=(x\cdot a)+(x\cdot (k+1))$\newline

Calculamos:

\begin{align*}
x\cdot(a+(k+1))& = x\cdot((a+k)+1)  \by {Pela associatividade da $+$}
& =(x\cdot(a+k))+x                  \by {Pela (m2) com $n:=x, \ m:=(a+k)$}
& = ((x\cdot a)+(x\cdot k))+x       \by {H.I}
& = (x\cdot a)+((x\cdot k)+x)  \by {Pela associatividade da $+$}
& = (x\cdot a)+(x\cdot (k+1))  \by {Pela (m2$^\leftarrow$) com $n:=x, \ m:=(k$}
\end{align*}

 \halmos
%fimdemonstração
 \newpage
 \section*{Problema 2.}
 Divulgando o LEM (princípio do terceiro excluido) tentei vender a idéia que seria essencial para demonstrar
mais proposições do que realmente é! Com as definições de par e ímpar seguintes
\begin{center}
$n \ par \stackrel{def}{\iff} (\exists k \in \ints)[n=2k]$ \\
$n \ impar \stackrel{def}{\iff} (\exists k \in \ints)[n=2k+1]$
\end{center}
e sem usar nenhum dos feitiços que discutimos (LEM, Reductio ad Absurdum, Lei da dupla negação, etc.)
demonstre diretamente (por indução) a proposição: todo número natural é par ou ímpar

%demonstração
\section*{Demonstração} 

Vou demonstrar a proposição.\newline
Por indução!

BASE: $(\exists t \in \ints)[0=2t] \ \stackrel{?}{ou} \ (\exists t \in \ints)[0=2t+1] $ 

Suponha t=0

Calculamos:
\begin{align*}
0& = 2\cdot0 \\
& = 0  
\end{align*}

Como $0\in \ints$, logo $(\exists t \in \ints)[0=2t] \ ou \ (\exists t \in \ints)[0=2t+1] $\\


PASSO INDUTIVO. Seja $k\in \ints$ tal que:

\begin{center}$(\exists t \in \ints)[k=2t] \ ou \ (\exists t \in \ints)[k=2t+1] $  (H.I)\end{center}

Basta demonstrar que $(\exists t \in \ints)[k+1=2t] \ ou \ (\exists t \in \ints)[k+1=2t+1] $ \newline

CASO  $(\exists t \in \ints)[k=2t]$

Seja $t\in \ints$, tal que $k=2t$

Calculamos:
\begin{align*}
k+1& = (2t)+1           \by {Hipótese do caso} 
& =2t+1   
\end{align*}

Como $t\in \ints$, logo $(\exists t \in \ints)[k+1=2t] \ ou \ (\exists t \in \ints)[k+1=2t+1] $\\

CASO  $(\exists t \in \ints)[k=2t+1]$

Seja $t\in \ints$, tal que $k=2t+1$

Calculamos:
\begin{align*}
k+1& = (2t+1)+1     \by {Hipótese do caso} 
& =2t+(1+1)         \by {Pela associatividade da $+$} 
& =2t+ 2         \\
& =2\cdot(t+1)         && \text{(distributividade da multiplicação)} 
\end{align*}
Como $(t+1)\in \ints$, logo $(\exists t \in \ints)[k+1=2t] \ ou \ (\exists t \in \ints)[k+1=2t+1] $
\halmos  \newpage
%fimdemonstração

\section*{Problema 3.}
Considere a funcção recursiva $\nats^2\longrightarrow \nats $, definida pelas equações:\\
  \begin{align*}
  \text{(K1)} && \alpha(0,x) &= x + 1 \\
  \text{(K2)} &&\alpha(n+1, 0) &= \alpha(n,1) \\
  \text{(K3)} && \alpha(n+1, x+1)&= \alpha(n, \alpha(n+1,x))
  \end{align*}\\

(i) Demonstre que para todo $x \in \nats, \alpha(1, x) = x + 2.$ \\
(ii) Dado que para todo $x \in \nats, \alpha(2, x) = 2x + 3$, demonstre que para todo $x \in \nats, \alpha(3, x) = 2^{x+3}-3$.
%demonstração
\section*{Demonstração}

Vou demonstrar a proposição (i).\newline
Por indução!

BASE: $\alpha(1,0)\stackrel{?}{=}0+2 $ 

Calculamos:


\begin{align*}
\alpha(1,0)& = \alpha(0,1)          \by {Pela (K2) com $n:=0$}
& =1+1                              \by {Pela (K1) com $x:=1$}
& = 2                               
\end{align*}

Passo Indutivo. Seja $k\in \ints$ tal que:

\begin{center} $\alpha(1,k)=k+2 $  (H.I)\end{center}

Basta demonstrar que $\alpha(1,(k+1))=(k+1)+2 $\newline

Calculamos:

\begin{align*}
\alpha(1,(k+1))& =\alpha(0,\alpha (1,k))    \by {Pela (K3) com $n:=0, x:=k$}
& =\alpha(0,(k+2))                          \by {H.I}
& = (k+2)+1                                 \by {Pela (K1) com $x:=k+2$}
& = k+(2+1)                                 \by {Associatividade da +}
& = k+(1+2)                                 \by {Comutatividade da +}
& = (k+1)+2                                 \by {Associatividade da +}
\end{align*}

 \halmos
%fimdemonstração

\newpage

\section*{Demonstração}

Vou demonstrar a proposição (ii).\newline
Temos que para todo $x \in \nats, \alpha(2, x) = 2x + 3$$_{(d13)}$\\
Por indução!

BASE: $\alpha(3, 0) = 2^{0+3}-3$

Calculamos:


\begin{align*}
\alpha(3,0)& = \alpha(2,1)          \by {Pela (K2) com $n:=2$}
& =2+3                              \by {d13 x:=1}
& = 5\\
& = 2^3-3
\end{align*}

Passo Indutivo. Seja $k\in \ints$ tal que:

\begin{center} $\alpha(3, k) = 2^{k+3}-3$  (H.I)\end{center}

Basta demonstrar que  $\alpha(3, (k+1)) = 2^{(k+1)+3}-3$ \newline

Calculamos:

\begin{align*}
\alpha(3,(k+1))& =\alpha(2,\alpha (3,k))    \by {Pela (K3) com $n:=2, x:=k$}
& =\alpha(2,(2^{k+3}-3))                          \by {H.I}
& = 2\cdot(2^{k+3}-3)+3                                 \by {d13 x:=$(2^{k+3}-3)$}
& =  (2\cdot2^{k+3})-6+3                                 \by {Propriedade distributiva }
& = (2\cdot2^{k+3})-3                                  \\
& = 2^{(k+3)+1}-3                                \by {Def. Potenciação}
& = 2^{k+(3+1)}-3                                \by {Associatividade da +}
& = 2^{k+(1+3)}-3                                \by {Comutatividade da +}
& = 2^{(k+1)+3}-3                                \by {Associatividade da +}
\end{align*}

 \halmos
%fimdemonstração
\pagestyle{fancy}.
\rfoot{Só isso mesmo.}
\end{document}
