\documentclass[a4paper]{article}
\usepackage[utf8]{inputenc}

\author{Dawerton Eduardo Carlos Vaz }


\usepackage{bbding}
\usepackage{natbib}
\usepackage{amsmath}
\usepackage{ amssymb}
\usepackage[T1]{fontenc}
\usepackage{fancyhdr}

\def\ints{{\mathbb Z}}
\def\nats{{\mathbb N}}
\def\by#1{&& \text{(#1)}\\}
\def\halmos{ \begin{flushright} \RectangleBold \end{flushright}}

\begin{document}


\section*{Problema 1.}
Conte em quantas maneiras podemos cobrir um tabuleiro de dimensão $2 \times n$ com peças-dominô (ou seja,
peças de dimensão $2 \times 1$).



%demonstração
\section*{Demonstração}

Seja D(n) o número de maneiras que podemos cobrir um tabuleiro de dimensão $2 \times n$ com peças-dominô.

Usando o principio da adição.
\begin{align*}
D(0)& = 1      &&    \\
D(1)& =1      &&   \\
\forall n>1 D(n)& =D(n-1) + D(n-2)
\end{align*}

$\underbrace{D(n-1)}_{\text{onde o tabuleiro começa com uma peça no vertical}}$

$\underbrace{D(n-2)}_{\text{{onde o tabuleiro começa com duas peças no vertical, ou horizontal}}}$

%fimdemonstração
 \newpage
 \section*{Problema 2.}
 Do alfabeto \{$a, b, c$\} desejamos formar strings de tamanho $\ell$ onde não aparece o substring ab. Em quantas
maneiras podemos fazer isso?

%demonstração
\section*{Demonstração} 
Seja K(n) o número de maneiras que podemos formar a string de tamanho n onde não aparece a substring ab.

Usando o principio da adição.
\begin{align*}
K(0)& = 1      &&    \\
K(1)& =3      &&   \\
K(2)& =8      &&   \\
\forall n>2 \ K(n)& =K(n-1) + K(n-1) + K(n-2) + K(n-2) - K(n-3) \\
\end{align*}

$\underbrace{K(n-1)}_{\text{onde $a$ string não começa com o caractere a}}$

$\underbrace{K(n-2)}_{\text{{onde $a$ string não começa com o caractere $ab$}}}$

$\underbrace{K(n-3)}_{\text{{onde a string começa com o caractere $b$ e depois é colocado o $a$ antes dele}}}$

\newpage
%fimdemonstração

\section*{Problema 3.}
Definimos a $\leq$ nos naturais assim:

\begin{center}
$n\leq m \stackrel{\text{def}}{\iff} (\exists k \in \nats)[n+k=m]$ \\
\end{center}
Demonstre por indução que $\leq$ é uma bem-ordem:\\ \
\begin{center}
Para todo $A \subseteq \nats$, $A = \emptyset $ ou $A$ possui mínimo.
\end{center}
Dica: visualise teu alvo assim:
\begin{center}
    $(\forall n \in \nats)(\forall A \subseteq \nats)[|A|=n \implies \text{A vazio ou possui minimo}]$
\end{center}

Onde $|A|$ denota a cardinalidade do $A$, ou seja, a quantidade de membros de $A$. 
Considere conhecido que:\\
\begin{align*}
|A|=0 & \iff A = \emptyset _{(a1101)}      &&    \\
|A|=Sn& \iff (\exists a \in A)(\exists A'\subseteq A)[a\notin A' \ \& \ |A'| = n] \ \& \ (\forall x \in A)[x=a \ \text{ou} \ x \in A']]  _{(b1101)}    &&
\end{align*}
Lembre-se que $m \in A$ é minimo membro de $A$ sse $(\forall x \in A)[m \leq a]$

%demonstração
\section*{Demonstração}

Vou demonstrar a proposição.\newline
Por indução!

BASE\\
Suponha |A|=0 \\
Logo pela $(a1101)$ $A = \emptyset $ ou $A$ possui mínimo.\\
Passo Indutivo. Seja $k\in \nats$ tal que:\\

\begin{center} $(\forall A \subseteq \nats)[|A|=k \implies A=\emptyset \ \text{ou possui minimo}]$  (H.I)\end{center}
Basta demonstrar que $(\forall A \subseteq \nats)[|A|=Sk \implies A=\emptyset \ \text{ou possui minimo}]$\\ \\
Seja $A \subseteq \nats \ \text{tal que} \ |A|=Sk$\\
Sejam $a\in A$ e $A' \subseteq A$ tais que $a\notin A' \ \& \ |A'| = n \ \& \ (\forall x \in A)[x=a \ \text{ou} \ x \in A']$\\
Logo $A' \subseteq \nats$ ($A' \subseteq A \subseteq \nats$)\\
Logo pelo H.I,  $A'= \emptyset$ ou possui minimo.\\
Dividindo em casos \\ \\
CASO $A'=\emptyset$\\ \\ 
Logo pela (b1101) $(\forall x \in A)[x=a]$\\
Logo $A$ possui minimo.\\ \\
CASO $A'$ possui minimo\\ 
Seja $y \in A'$ tal que $(\forall m \in A')[y \leq m]$\\
Logo $(\forall m \in A)[y \leq m\leq a \ \text{ou} \ a \leq y\leq m]$\\
Logo A possui minimo.
 \halmos
%fimdemonstração

\newpage
\section*{Problema 4.}
Faria sentido trocar o ‘$(\exists a \in A)$’ por ‘$(\forall a \in A)$’ na penúltima linha do Problema 3? Explique curtamente.\\
Resposta:
Não, pois $A'$ não poderia ser subconjunto de $A$ \\ \\




%fimdemonstração
\pagestyle{fancy}.
\rfoot{Só isso mesmo.}
\end{document}
